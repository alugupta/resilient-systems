\section{Strengths and Weaknesses of Existing Works}
\label{sec-discussions}

A number of studies have been discussed for both memory encryption and
integrity verification. This section will compare the discussed techniques for
encryption and integrity verification separately.

\subsection{Memory Encryption}
The following table summarizes the studies discussed for memory encryption
mechanisms \cite{suh-memIntEnc, aegis, gcmMem, duece} for parameters such as
encryption algorithms used, memory isolation capabilities, hardware overhead,
power and durability.

\begin{center}
    \textbf{Table 1. Comparing Memory Encryption Mechanisms}
  \resizebox{\columnwidth}{!}{

    \begin{tabular}{| c | c | c | c | c |}
      \hline
      & \textbf{AES-CTR} & \textbf{AEGIS} & \textbf{DUECE} & \textbf{GCM} \\ \hline

     Encryption Alg. & AES-CTR & AES-CTR             & AES-CTR              & GCM  \\ \hline
     Mem. Isolation  & No      & {\color{green} Yes} & No                   & No   \\ \hline
     Counters        & 1       & 1                   & 1                    & 1    \\ \hline
     AES-Engines     & 1       & 1                   & {\color{red} 2}      & 1    \\ \hline
     Avg. Bit Flips  & 50\%    & 50\%                & {\color{green} 24\%} & 50\% \\ \hline
     Durability      & 1x      & 1x                  & {\color{green} 2x}   & 1x   \\ \hline

    \end{tabular}
  }
\end{center}
The AES-CTR based method \cite{suh-memIntEnc} acts as the baseline approach for
comparing memory encryption mechanisms as it provides the minimum required
security features without tremendous hardware enhancements. AEGIS \cite{aegis}
leverages the same cryptographic encryption technique as the AES-CTR
\cite{suh-memIntEnc} study, however it also provides memory isolation between
users and supervisor. AEGIS is the only study to use secure OS-kernel level
techniques to provide additional memory confidentiality protection.

DUECE \cite{duece} is the only study to optimize writing encrypted data to
memory in terms of performance, power and memory durability. DUECE reduces the
average bit flips for programs with sparse, regular memory access patterns to
24\% percent from the nominal 50\% of encrypted data as determined by the
\TT{Avalance} effect \cite{avalance}. The reduction greatly improves write
performance, write bandwidth, power consumption and durability. DUECE is also
the only study to leverage \textit{wear leveling} mechanisms to improve memory
durability by a factor of 2x - especially for non-volatile memory systems such
as PCM. Though DUECE uses both the LCTR and TCTR, since it only stores one
counter in memory, since the TCTR is derived from the LCTR, we say that it only
has 1 counter worth of memory overhead. DUECE however does require two
AES-Engines to efficiently decrypt the ciphertext on read operations
\cite{duece}. Finally, even though the GCM approach may in reality incur a
lower memory overhead compared to the DUECE approach, we omit comparing
specific bitwidths for counters as these are largely system and design
dependent and obfuscate the general trade-offs of each approach.

In general, the DUECE based memory encryption technique is the most optimized
for performance, power, and durability however there may be room for
improvement to reduce the number of required AES-Engines and provide additional
virtual memory isolation techniques.

\subsection{Integrity Verification}
The following table summarizes the studies discussed and their general
trade-offs in regards to integrity verification of off-chip memories
\cite{patTree} \cite{merkle}  \cite{multicoreEnc} \cite{suh-memIntEnc}
\cite{aegis} \cite{gcmMem}.

For memory integrity verification, the Merkle trees \cite{merkle} serve the as
the baseline protection technique since the latter improve upon Merkle trees in
different ways.  We do not compare memory overhead for each technique as the
studies do not employ similar systems or use comparable metrics for clear
overhead comparisons.

\begin{center}

  \textbf{Table 2. Comparing Integrity Verification Mechanisms}

  \resizebox{\columnwidth}{!}{
    \begin{tabular}{| c | c | p{1.5cm} | c | c |}
      \hline
      & \textbf{Parallel} & \textbf{Mem. Isolation} & \textbf{MultiCore} & \textbf{Alg.} \\ \hline

     Merkle    & No                  & No                  & No                  & Hash       \\ \hline
     PAT       & {\color{green} Yes} & No                  & No                  & MAC        \\ \hline
     Log Hash  & No                  & No                  & No                  & Hash       \\ \hline
     AEGIS     & No                  & {\color{green} Yes} & No                  & Hash       \\ \hline
     GCM       & No                  & No                  & No                  & GMAC       \\ \hline
     MultiCore & No                  & No                  & {\color{green} Yes} & Hash       \\ \hline

    \end{tabular}
  }
\end{center}

As seen in Table 2, each technique exhibits it's own advantages and
disadvantages. PATs are the only fully parallelizable authentication trees
covered in this study. Generally speaking however, PAT has a greater memory
overhead than Merkle trees - by a factor of $\frac{3}{2}$ \cite{surveyInt}.
Similarly, AEGIS is the only system that offers memory isolation between users
and the supervisor and Yan et. al.'s authentication mechanism is the only one
that supports multicores. AEGIS requires secure OS-kernel support to provide
memory isolation between users and the supervisor. The multicore approach
\cite{multicoreEnc} adds additional hardware complexity to mediate multiple
read/write operations and keeping a consistent memory state. The Log Hash
\cite{suh-memIntEnc} approach is more efficient for programs where
\TT{integrity check} operation calls are infrequent. Finally the GCM approach
\cite{gcmMem} leverages a more efficient cryptographic authentication algorithm
- \TT{GMAC/GHASH} than the other studies. Comparable to hiding the decryption
latency on read operations by using AES-CTR generated \TT{OTP}s,
\TT{GMAC/GHASH} can also hide the \TT{integrity check} operation by
pre-generating \TT{OTP}s which are used to authenticate the data. This greatly
improves the latency of authenticating memory.

In general, there is no comprehensive memory authentication technique like
DUECE for memory encryption that provides a holistic solution to integrity
verification.
