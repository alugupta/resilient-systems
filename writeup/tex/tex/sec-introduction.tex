\section{Problem Statement and Motivation}
\label{sec-introduction}

Recently, there has been an increasing effort in developing trusted computing
platforms augmented with specialized hardware modules that provide security
features such as authentication and decryption/encryption. One such security
feature that remains a focus in trusted computing research is protecting
off-chip memory. Protecting off-chip memory includes integrity verification and
maintaining confidentiality of private data. Challenges in designing memory
protection mechanisms include providing memory authentication and encryption
primitives at a low cost without compromising security. Current work also
focuses on memory protection schemes for general purpose and high performance
computing systems. The domain of memory protection in the mobile and embedded
computing domains is relatively less studied and pose interesting challenges.
Mobile and embedded devices often operate under strict power and area
constraints. Providing secure memory protection while following the set power
and area constraints as well as maintaining performance of the computing
devices is an ongoing challenge.

As mobile and embedded computing chips incorporate an increasing number of
transistors, they inevitably become more heterogeneous - encompassing
multi-core processors, graphics processing units (GPUs) and hardware
accelerators. Providing memory authentication and encryption for such
heterogeneous, shared memory systems is another challenge as current work
mainly focuses on the uni-processor architectures.

This survey aims to explore state-of-the-art architectures and techniques
developed by research in the area of efficient memory integrity verification
and encryption. We limit this survey to mainly exploring hardware based memory
protection schemes as past research as shown that implementing the encryption
and integrity verification schemes solely in software incurs large processing
overhead. This survey does however, discuss some memory protection schemes that
couple secure OS-kernel features and hardware modules for providing memory
protection.

\subsection{Thread Model}
This survey assumes the following threat model, unless stated otherwise, in the
following discussions:

\begin{enumerate}[noitemsep, topsep=0pt]
  \item Attackers are capable of constructing \TT{spoofing} devices or hardware.
  \item Attackers are capable of \TT{snooping} any interconnection fabrics and
    buses connecting the processors and off-chip memory. This involves both
    software and hardware based snooping attacks.
  \item Attackers are capable of \TT{replay} attacks. Replay attacks also
    include intelligent replay attacks where attacks are made after monitoring
    the communication between the processor and off-chip memory. We assume that
    any communication patterns may be exploited in this threat model.
  \item Attackers are capable of isolating the off-chip memory chip and reading
    and modifying the contents of the memory. This is especially pronounced in
    systems where non-volatile memory technologies, such as phase change memory
    (PCM), are used since memory persists even after the system is powered off.
    When using non-volatile memory technologies, limited write durability and
    must also be considered.
\end{enumerate}

Generally, the thread model assumes that the interconnect fabric and buses are
not secure and attackers have access and the capability to tamper with these
hardware systems. The security measures employed aim to prevent or detect
attacks made under this threat model. Attacks targetting availability of
hardware modules such as the processor or off-chip memory are outside of the
scope of this survey.

The following sections are organized as follows: Section
\ref{sec-related} reviews the related work in the research domain, Section
\ref{sec-discussions} discusses the strengths and weakness of the current work
and Section \ref{sec-conclusions} concludes with the future directions for
low-cost memory protection techniques.
