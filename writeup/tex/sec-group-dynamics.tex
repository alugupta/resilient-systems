\section{Group Dynamics}
\label{sec-group-dynamics}

%Describe contributions by each group member / how was the work split between
%the two.
Generally, we worked together for all parts of the project. To start with, we
came up with the design and overall experiment together. This included
investigating Gem5, SESC, and Pin. Since Monica had more experience regarding
the computer architectural simulators she was able to better understand the
benefits of using Pin versus Gem5 and SESC. Next we split up the work for
getting Pin setup, acquiring benchmarks and investigating DRAM simulators.
Monica worked on acquiring the benchmarks and finding DRAM simulators - namely
DRAMSim. Udit worked on organizing, understanding and implementing a working
Pintool for generating the memory trace. Once the initial Pintool for
generating a memory trace using the \TT{pinatrace} and \TT{safecopy} examples,
(without a data-cache) was completed, we worked together to test the Pintool on
small generic \TT{bash} commands and sample programs we wrote.

The two of us then worked together on understanding DRAMSim and understanding
how to run the infrastructure using user inputted memory traces. After a few
days of analyzing the tool flow, the generated power traces, and the source
code the two of us decided that DRAMSim would not be a viable option for our
project. At this point we went back to existing literature to find simple DBI
models that we could use for power modeling. This is when we found Hollis'
model \cite{hollis} and agreed to use it for our study. Concurrently we worked
on implementing a data-cache instrumentation tool using Pin to isolate the
cache misses - a more realistic configuration for our project. Finally, once
the DBI model was decided, Udit focused on writing the simple Python script to
analyze the memory traces and Monica focused on organizing all of our sources,
background research and began working on the power-point.

Needless to say, most of the project was done together since much of the work
was understanding existing models and tools.
