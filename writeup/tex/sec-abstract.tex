\begin{abstract}
\label{sec-abstract}
%A short summary of the paper
As the number and heterogeneity of computing devices steadily increases, it
becomes more crucial for hardware and software designers to provide security
primitives to users. Each day, these devices communicate countless pieces of
sensitive data that must be adequately protected. The need to maintain the
confidentiality of memory is just one aspect of this security problem, as it
safeguards a user's locally stored information. Recent research efforts have
focused on developing cryptographic memory encryption primitives for the
general purpose and high performance computing domains however mobile and
embedded devices pose an interesting challenge with strict first-order power
constraints. This study aims to model the effect of memory encryption on
off-chip memory power consumption by focusing on the effects of link
termination and signal reflections that arise when driving Data-Bus Inversion
(DBI) enabled DDR4 memory technologies. Using MiBench, a set of applications
representative of mobile workloads, and Pin, a computer architectural
simulator, we find that memory encryption does incur a significant power
overhead. We provide a framework for users to run arbitrary binary executables,
specify first-level cache configurations and analyze DBI ratios of secure
memory systems with encryption and insecure memory systems without encryption.
\end{abstract}
