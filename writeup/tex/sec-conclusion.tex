\section{Conclusion}
\label{sec-conclusions}

Generally our study hopes to make a small step in developing a reliable model
for power consumption overheads of secure memory systems. Where previous
studies have either focused on performance and area or non-volatile memory
systems, our system leverages traditional DDR based memory technologies ---
namely DDR4. Focusing on DBI-enabled memory systems, we have developed a
simulation infrastructure that can be used to model the affects of encrypted
data on ODT power consumption resulting from link termination and signal
reflections. In theory, this model could be used for a wide array of I/O
devices that support DBI, or similar schemes. We leverage Pin, a dynamic
instrumentation tool for computer architecture simulations and MiBench as a
representative set of mobile workloads. We find that encryption, as expected,
does incur a significant power overhead on DBI-enabled off-chip memory systems.

In the future we hope to develop a real computing system with a DBI-enabled
off-chip memory system and verify our models and collect empirical data from
the prototype. This would would also enable researchers to measure system-wide
power metrics for improving secure processor designs. We also hope to
investigate alternative write schemes to DBI --- especially ones targeting
encrypted read and write operations. Finally, we hope that one day a single
unified model or infrastructure for simulating the performance, area and power
consumption of real mobile computing systems is developed to facilitate more
productive design space explorations of secure mobile computing.
