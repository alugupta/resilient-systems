\section{Future Directions}
\label{sec-conclusions}

This survey aims to summarize the current work in efficient low-cost memory
protection - integrity verification and decryption/encryption. This survey is a
brief overview of the field and thus does not cover all techniques and
approaches to memory protection. In stead, this survey uses representative
studies to broadly illuminate the domain. From the studies it seems as though
more thorough work has been conducted in memory encryption - namely the DUECE
approach. Integrity verification still needs more system level development -
optimizing for power, performance, and area overhead of homogeneous uni-core
and multicore processors and well as heterogeneous systems. It is clear that
simple uni-core processor protection schemes do not directly apply to multicore
systems. Much of the current research also does not optimize a unified
encryption and integrity verification systems. Studies usually focus on one of
these memory protection domains which raises the question if there is a less
explored design space for unified memory protection mechanisms. Also, though
some studies consider the performance and memory durability, this is also less
explored than performance benchmarks for various encryption and integrity
verification schemes.

Some avenues of future work include exploring heterogeneous architecture memory
protection mechanisms, memory protection mechanisms with power and area as
first-order constraints - focusing on the domain of embedded or \TT{Internet of
Things} domains, or heterogeneous architectures. There may also be merit in
exploring more GCM based encryption and integrity verification schemes as the
cryptographic authenticated encryption algorithm provides efficient derivatives
of both security features.
