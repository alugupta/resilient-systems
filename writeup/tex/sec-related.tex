\section{Related Work}
\label{sec-related}

%Describe related work. You need to show how the previous work is related to or
%different from your project.
Generally, previous work in the domain of secure memory protection has focused
on the performance and area overhead of cryptographic memory encryption ---
especially for general purpose and high performance computing systems. For
instance, Suh et. al. \cite{suh-memIntEnc} explores efficient memory encryption
and integrity verification schemes using logs to maintain snapshots of the
memory. Generally, Suh et. al. \cite{suh-memIntEnc} expand on the same memory
encryption scheme shown in Figure \ref{fig:sys-arch} by using the AES-CTR
operating mode and encryption not only the data value but meta-data information
such as the address, a fixed value and counter for increased protection.

Previous work has also explored the more efficient cryptographic encryption
schemes such as Galois Counter Mode (GCM) \cite{nistGCM} for memory encryption.
For instance, Yan et. al. \cite{gcmMem} leverage GCM's high degree of
parallelism to provide memory encryption and integrity verification. Yan et.
al. and Suh et. al. both focus on providing both memory encryption and
integrity verification --- suggesting that there may be a relatively unexplored
design space for efficient systems in terms of performance, area, power and
design complexity that offers both security features.

Recent studies have begun to explore power efficiency of secure memory systems.
For instance, Young et. al. \cite{duece} explore memory encryption for
non-volatile memory systems focusing on power and durability. Young et. al.
propose DEUCE - a write efficient encryption scheme that maintains two virtual
counters for encrypting memory. DEUCE's power model is dependent on minimizing
the number of bit flips by using write compression schemes such as
\TT{Flip-n-Write} \cite{fnw}. Similar to minimizing the DBI-AC ratio, DEUECE
was able to minimize the number of bit flips from 50 \% on average to 24 \% by
clever cache-line encryption schemes with dual counters.

Existing research efforts have focused on performance and area overhead of
memory encryption. Though DEUECE is one example of researchers considering
power consumption in regards to memory encryption, the study focuses on systems
using non-volatile memory. Few studies explore encryption for off-chip DDR
based memory technologies in terms of empirical results and model based
simulations. We believe that this study is a small step in modeling the power
consumption of secure mobile systems.
